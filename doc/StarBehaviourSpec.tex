\documentclass[11pt]{article}

\setlength{\oddsidemargin}{0in}
\setlength{\evensidemargin}{0in}
\setlength{\topmargin}{-0.5in}
\setlength{\headheight}{0pt}
\setlength{\topskip}{0pt}
\setlength{\textwidth}{6.5in}
\setlength{\textheight}{9in}
%\setlength{\parindent}{0pt}
\setlength{\parskip}{1mm}

\usepackage{epsf}

% Magic incantation to allow proper PostScript to PDF conversion:
  \usepackage[T1]{fontenc} % type 1 fonts
  \usepackage{times}       % Times-Roman

\sloppy

\newcommand{\Fwname}{Johar}

\begin{document}

\begin{center} \bf \Large
Johar ``Star'' Interface Interpreter \\
Requirements Specification -- Behaviour
\end{center}

\section{Top-Level Behaviour}

\begin{enumerate}
\item Read the IDF.
\item Create the Main Panel.
  \begin{enumerate}
  \item Every {\tt CommandGroup} should correspond to a menu on the
    menu bar.
  \item The menu item for a command should be the {\tt Label}
    of the command, followed by ``{\tt ...}'' if selecting the 
    command will result in a Command Window being created (see below).
  \end{enumerate}
\item Initialize the Text Area to empty.
\item Create the GemSetting and validate it.
\item Call the application engine's initialization method.
\item Refresh the tables (see below).
\item Append a horizontal line (this could be just a line of minus
  characters) to the bottom of the Text Area.  (This will separate any
  initial message from the application engine from the messages that are
  responses to the commands.)
\item Call the {\it isActive} method of every command, and make each menu
  item of each menu active or inactive (greyed out) according to the
  result of the {\it isActive} method of the command.
\end{enumerate}

\section{Selecting a Command from a Menu}

When the user selects a command from a menu:
\begin{enumerate}
\item If any table selection is incomplete for the command,
  then pop up a dialog box with an ``OK'' button telling the user
  which tables they have to select rows in in order to issue the
  command.
\item Otherwise, if no stage in the command has a queryable parameter
  (i.e., there are no parameters to any of the stages of the command, or all
  parameters in all stages of the command are
  {\tt tableEntry} parameters corresponding to browsable tables):
  \begin{enumerate}
  \item Call {\tt gemSetting.selectCurrentCommand(cmdName)}.
  \item Call {\tt gemSetting.selectCurrentStage(0)}.
  \item For each stage \verb/i/ in the command, perform the Stage Loop
    from the IntI Specification document.
  \item Call the application engine method.
  \item Execute the Command Wrapup Procedure with parameter {\tt false}
    (indicating command was not cancelled).
  \end{enumerate}
\item Otherwise (i.e., if no table selection is incomplete for
  the command, but there are queryable parameters):
  \begin{enumerate}
  \item Call {\tt gemSetting.selectCurrentCommand(cmdName)}.
  \item Create the Command Dialog for the command.
    (The Command Dialog will take over the rest of the processing
    of the command.)
  \end{enumerate}
\end{enumerate}
A table selection is {\it incomplete for a given command}
if it contains at least one stage with a parameter of type
{\tt tableEntry}, such that the {\tt MinNumberOfReps} for that
parameter is greater than the number of rows that are selected
in the {\tt SourceTable} of the parameter, and the parameter
does not have a {\tt ParentParameter}.

\section{Command Wrapup Procedure}

The Command Wrapup Procedure is called after the user selects a
command from the menu and some processing has been done.  It takes
one parameter, called {\tt commandWasCancelled}.
\begin{enumerate}
\item If there is a Command Dialog, delete it.
\item If {\tt commandWasCancelled} is false (i.e., the application
  engine method corresponding to the command has been called):
  \begin{enumerate}
  \item Determine whether the application should quit, by checking
    the {\tt QuitAfter} attribute and/or calling the {\tt QuitAfterIf}
    method of the command.
  \item If the application should quit:
    \begin{enumerate}
    \item If {\tt lastDisplayedText} is not the empty string, then
      create a dialog box containing {\tt lastDisplayedText} and
      an ``OK'' button, and display it.  Wait for the user to
      click ``OK'', and then delete the dialog box.
    \item Exit the application, e.g. using {\tt System.exit(0)}.
    \end{enumerate}
  \item Otherwise:
    \begin{enumerate}
    \item Append a horizontal line to the Text Display Area, in order to
      separate the last command's output from the output of any
      future commands.
    \item Refresh the tables (see below).
    \end{enumerate}
  \end{enumerate}
\item Call the {\it isActive} method of every command, and make each
  menu item of each menu active or inactive (greyed out) according to
  the result of the {\it isActive} method of the command.
\end{enumerate}

\section{Refreshing the Tables}

To refresh the tables:
\begin{enumerate}
\item For each table currently being displayed in the Table Area
  that is now hidden, delete the corresponding tab.
\item For each table not currently being displayed in the Table
  Area that is now non-hidden, create a tab in the Table Area.
  (After this happens, it should be the case that the only tables
  with tabs in the Table Area are non-hidden tables.)
\item For each non-hidden table which has been updated since the
  last command execution terminated, update the data on the tab
  to reflect the contents of the table.
\item If there is a top table now, place that table's tab on top.
\end{enumerate}

\section{The {\tt ShowTextHandler}}

The ShowTextHandler for Star handles text according to the prominence of
the text.
\begin{enumerate}
\item 0-1999:  For each line of text in the message:
  \begin{enumerate}
  \item Truncate the line of text, if necessary, to fit in the Status Bar.
  \item Set the Status Bar to the text.
  \end{enumerate}
  [Note: This will cause the last line of the most recent message to
  overwrite anything that was in the Status Bar before.  This is OK
  because these messages are ``low prominence''.]
\item 2000-2999:  Append the text to the text being displayed in the
  Text Display Area.  Append the text also to {\tt lastDisplayedText}.
\item 3000 and higher:  Create a dialog box containing the text and an
  ``OK'' button, and display it.  When the user clicks the ``OK'' button,
  the box should be deleted.
\end{enumerate}

\section{The Command Dialog Box}

\paragraph{Creating the command dialog box:}
\begin{enumerate}
\item Call {\tt gemSetting.selectCurrentStage(0)}.
\item Perform an Initialize Stage procedure.
\item While the current stage has no queryable parameters, perform
  a Next Stage procedure (see below).
\item Update the dialog box to reflect the current stage, as described
  in the Star GUI Specification document.
\end{enumerate}

\paragraph{When the user presses the Cancel button:}
\begin{enumerate}
\item Perform the Command Wrapup procedure, with the parameter {\tt true}.
\end{enumerate}

\paragraph{When the user presses the Previous button (if it is not greyed out):}
\begin{enumerate}
\item Perform a Previous Stage procedure (see below).
\item While the current stage has no queryable parameters, perform
  a Previous Stage procedure (see below).
\item Update the dialog box to reflect the current stage, as described
  in the Star GUI Specification document.
\end{enumerate}

\paragraph{When the user presses the Next button (if it is not greyed out):}
\begin{enumerate}
\item Perform a Next Stage procedure (see below).
\item While the current stage has no queryable parameters, perform
  a Next Stage procedure (see below).
\item Update the dialog box to reflect the current stage, as described
  in the Star GUI Specification document.
\end{enumerate}

\paragraph{When the user presses the OK button (if it is not greyed out):}
\begin{enumerate}
\item Perform the Wrap Up Stage procedure.
\item For each stage \verb/i/ in the current command, from stage 0 to the
  last stage:
  \begin{enumerate}
  \item Call {\tt gemSetting.selectCurrentStage(i)}.
  \item Perform the Initialize Stage procedure.
  \item Perform the Wrap Up Stage procedure.
  \item If the Wrap Up Stage procedure returns false, then return.
  \end{enumerate}
  (Rationale:  there may have been some previous stages which have
  become invalid as a result of the current stage; also, there may be
  stages with non-queryable parameters whose parameter values have never
  been loaded.)
\item Call the command method.
\item Perform the Command Wrapup procedure, with the parameter false
  (indicating the command was not cancelled).
\end{enumerate}

\paragraph{Previous Stage procedure:}
\begin{enumerate}
\item Perform a Wrap Up Stage procedure.
\item If the Wrap Up Stage procedure returns true:
  \begin{enumerate}
  \item If the current stage is \verb/i/, set the current stage to \verb/i-1/.
  \item Call {\tt gemSetting.selectCurrentStage(i)}.
  \item Perform an Initialize Stage procedure (see below).
  \end{enumerate}
\end{enumerate}

\paragraph{Next Stage procedure:}
\begin{enumerate}
\item Perform a Wrap Up Stage procedure.
\item If the Wrap Up Stage procedure returns true:
  \begin{enumerate}
  \item If the current stage is \verb/i/, set the current stage to \verb/i+1/.
  \item Call {\tt gemSetting.selectCurrentStage(i)}.
  \item Perform an Initialize Stage procedure (see below).
  \end{enumerate}
\end{enumerate}

\paragraph{Initialize Stage procedure:}
\begin{enumerate}
\item If the current is a stage that has never been initialized yet, then
  for each parameter in the current stage:
  \begin{enumerate}
  \item If the parameter has a default value, set the value of the
    parameter in the CommandModel to the default value.
  \item Otherwise, if the parameter has a DefaultValueMethod, then call it
    and set the value of the parameter in the CommandModel to the default
    value.
  \end{enumerate}
\end{enumerate}

\paragraph{Wrap Up Stage procedure:}
\begin{enumerate}
\item Validate the current values of the current repetitions of the
  parameters, as indicated in requirements 26-51 of the IntI
  specification.
\item Load the current values of the current repetitions of the parameters
  for the current stage into the Gem.
\item Call the ParameterCheckMethod of the current stage, if it has one.
\item If the ParameterCheckMethod exists and returns a non-null, non-empty
  string:
  \begin{enumerate}
  \item Show the user the string in a popup with an ``OK'' button.
  \item When the user clicks OK, if the current stage has no queryable
    parameters, then perform the Command Wrapup procedure, with the
    parameter {\tt true}.  (Rationale:  This might happen in some obscure
    situations, such as when a {\tt tableEntry} parameter has a parent
    parameter in another stage which gets set to the parent value.  In
    these situations, there is nothing we can do but cancel the command.)
  \item Return false.
  \end{enumerate}
\item Otherwise, return true.
\end{enumerate}

\section{The Parameter Section}

\paragraph{When the user clicks Add Another:}

\paragraph{When the user clicks Move Up:}

\paragraph{When the user clicks Move Down:}

\paragraph{When the user clicks Delete:}

\section{The Main Panel}

See Figure \ref{mainPanelFig}.  The Main Panel is displayed when the
application is started, and remains visible throughout the lifetime of the
application.
\begin{enumerate}
\item At the top is the {\it Menu Bar}.  This shows one menu
  for each command group in the application.
\item Each menu is labelled in the Menu Bar by the name of the
  command group.
\item The menu items in each menu are the commands that are in
  that command group.
\item A menu item is clickable (not greyed out) if the corresponding
  command is {\it active}.
\item A command is {\it active} at a given point if either it has no
  {\tt ActiveIfMethod}, or it has an {\tt ActiveIfMethod} and that
  method returns {\tt true}.
\item On the left is the {\it Text Display Area}.  This shows all
  text messages sent by the application engine using {\tt displayText}
  that are not displayed through some other means.
\item On the right is the {\it Table Area}.  This is a tabbed pane.
\item In the Table Area, there is one tab corresponding to every
  revealed Table currently in the application.
\item The application decides which Tables to conceal and reveal.
\item If, during the processing of a command, the application engine sets
  the top table, then immediately after the command has finished
  processing, the topmost table in the Table Area is the last table to
  have been set as the top table.
\item If, during the processing of a command, the application engine
  does not set the top table, then immediately
  after the command has finished processing, the topmost table in the
  Table Area remains unchanged from the time that the command was issued.
\item Notwithstanding the table set as the topmost table by the app
  engine, the user can (between issuing one command and issuing the
  next command) click on a table tab in order to move that table to
  the top.
\item At the bottom is the {\it Status Bar}.  This shows low-prominence
  messages sent by the application engine using {\tt displayText}.
\end{enumerate}

\begin{figure}[t]

\centerline{\epsffile{StarGui_mainPanel.eps}}

\caption{
  The main panel.
}
\label{mainPanelFig}
\end{figure}


\section{The Command Dialog Box}

\begin{figure}

\centerline{\epsfxsize=5.5in \epsffile{StarGui_commandDialogBox.eps}}

\caption{
  The command dialog box.
}
\label{commandDialogBoxFig}
\end{figure}

See Figure \ref{commandDialogBoxFig}.
\begin{enumerate}
\item At the top is the {\tt Label} attribute of the command.
\item Under that is one section for every {\it queryable parameter}
  (see below) in the current stage.
\item Many commands have only one stage.  However, if there is
  more than one stage, only the queryable parameters in the current stage
  will appear in the parameter section.
\item There is one section of the box for each queryable parameter in the
  current stage.
\item At the bottom are two to four buttons: {\tt Cancel}, {\tt Previous},
  {\tt Next}, and {\tt OK}.  {\tt Cancel} and {\tt OK} always appear;
  {\tt Previous} and {\tt Next} do not always appear.
\item The {\it Cancel button} always appears and is always enabled.
\item The Cancel button is on the left-hand side of the dialog box.
\item The {\it Previous button}:
  \begin{enumerate}
  \item Appears if there is more than one {\it queryable stage} (see below).
  \item Is enabled if the current stage is not the first queryable stage.
  \end{enumerate}
\item The {\it Next button}:
  \begin{enumerate}
  \item Appears if there is more than one queryable stage.
  \item Is enabled if the current stage is not the last queryable stage.
  \end{enumerate}
\item The {\it OK button}:
  \begin{enumerate}
  \item Always appears.
  \item Is enabled if there are no {\it incomplete stages} (see below).
  \end{enumerate}
\item The OK button is on the right-hand side of the dialog box.
\item A parameter is {\it inactive} if both (a) it has a {\tt ParentParameter},
  and (b) the current value of the {\tt ParentParameter} is not the
  {\tt ParentValue} for this parameter.
\item A {\it queryable parameter} is any parameter
  that is not a {\tt tableEntry} parameter with
  a browsable {\tt SourceTable}.
\item A {\it queryable stage} is a stage that contains some
  queryable parameter.
\item An {\it incomplete stage} is a stage that contains some parameter
  such that: (a) the parameter is not inactive, (b) there is no
  {\tt DefaultValue} or {\tt DefaultValueMethod} for the parameter,
  and (c) the current number of repetitions is less than the
  {\tt MinNumberOfReps} for the parameter.
\end{enumerate}

\section{Parameter Section of the Command Dialog Box}

\begin{figure}

\centerline{\epsfxsize=5.5in \epsffile{StarGui_parameterSection.eps}}

\caption{
  The parameter section of the command dialog box.
}
\label{parameterSectionFig}
\end{figure}

Every queryable parameter in the current stage is represented by a section
of the command dialog box.  See Figure \ref{parameterSectionFig}.
\begin{enumerate}
\item On the left is the {\tt Label} attribute of the parameter,
  at the top of the left-hand side, followed by a colon.
\item On the right is one subsection for each repetition of the parameter,
  and sometimes a small {\it Add Another} button.
\item Initially (before any user interaction), the number of repetition
  sections will be equal to maximum(1, $m$), where $m$ is the value of
 {\tt MinNumberOfReps} for the parameter.
\item The {\tt Add Another} button:
  \begin{enumerate}
  \item Has the ToolTip ``Add another''.
  \item Is a small button with a plus sign (``{\tt +}'') in it.
  \item Appears only if {\tt MinNumberOfReps} is not equal to
    {\tt MaxNumberOfReps} for this parameter.
  \item Is enabled only if the current number of repetitions of the
    parameter is less than {\tt MaxNumberOfReps} for this parameter.
  \item Is on the left-hand side of the right-hand half of the parameter
    section.
  \end{enumerate}
\item The entire section corresponding to the parameter is greyed out
  if the parameter is inactive.
\end{enumerate}

\section{Repetition Section of the Parameter Section}

\begin{figure}

\centerline{\epsfxsize=5.5in \epsffile{StarGui_repetitionSection.eps}}

\caption{
  The repetition section of the parameter section.
}
\label{repetitionSectionFig}
\end{figure}

Every repetition of a queryable parameter in the current stage is
represented by a section of the command dialog box.  See Figure
\ref{repetitionSectionFig}.
\begin{enumerate}
\item On the left is a widget which the user can use to select the
  value of the repetition.  This widget will be different for different
  types of parameters.  For instance, for a {\tt boolean} parameter,
  it may be a widget consisting of two radio buttons labelled ``Yes''
  and ``No'', whereas for a {\tt string} parameter it may be a
  text area.
\item On the right of the section is zero to three small buttons
  with icons in them.
\item The {\tt Move Up} button:
  \begin{enumerate}
  \item Is a small button with an up-arrow in it.
  \item Has the ToolTip ``Move up''.
  \item Appears only if
    {\tt MaxNumberOfReps} is not equal to 1 for this parameter, and
    {\tt RepsModel} for this parameter is {\tt sequence}.
  \item Is enabled only if this is not the first repetition (the topmost
    repetition).
  \end{enumerate}
\item The {\tt Move Down} button:
  \begin{enumerate}
  \item Is a small button with an down-arrow in it.
  \item Has the ToolTip ``Move down''.
  \item Appears only if
    {\tt MaxNumberOfReps} is not equal to 1 for this parameter, and
    {\tt RepsModel} for this parameter is {\tt sequence}.
  \item Is enabled only if this is not the last repetition (the bottommost
    repetition).
  \end{enumerate}
\item The {\tt Delete} button:
  \begin{enumerate}
  \item Is a small button with an X (``\verb/x/'') in it.
  \item Has the ToolTip ``Delete''.
  \item Appears only if {\tt MinNumberOfReps} is not equal to
    {\tt MaxNumberOfReps} for this parameter.
  \item Is enabled only if the current number of repetitions of the
    parameter is more than maximum(1, $m$), where $m$ is the value
    of {\tt MinNumberOfReps} for this parameter.
  \end{enumerate}
\end{enumerate}


\end{document}
