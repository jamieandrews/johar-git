\documentclass[11pt]{article}

\setlength{\oddsidemargin}{0in}
\setlength{\evensidemargin}{0in}
\setlength{\topmargin}{-0.5in}
\setlength{\headheight}{0pt}
\setlength{\topskip}{0pt}
\setlength{\textwidth}{6.5in}
\setlength{\textheight}{9in}
%\setlength{\parindent}{0pt}
\setlength{\parskip}{1mm}

\usepackage{epsf}

% Magic incantation to allow proper PostScript to PDF conversion:
  \usepackage[T1]{fontenc} % type 1 fonts
  \usepackage{times}       % Times-Roman

\sloppy

\newcommand{\Fwname}{Johar}

\begin{document}

\begin{center} \bf \Large
Johar ``Star'' Interface Interpreter \\
Requirements Specification -- Behaviour
\end{center}

\section{Top-Level Behaviour}

\begin{enumerate}
\item Read the IDF.
\item Create the Main Panel.
  \begin{enumerate}
  \item Every {\tt CommandGroup} should correspond to a menu on the
    menu bar.
  \item The menu item for a command should be the {\tt Label}
    of the command, followed by ``{\tt ...}'' if selecting the 
    command will result in a Command Dialog Box being created (see below).
  \end{enumerate}
\item Initialize the Text Area to empty.
\item Create the GemSetting and validate it.
\item Call the application engine's initialization method.
\item Refresh the tables (see below).
\item Append a horizontal line (this could be just a line of minus
  characters) to the bottom of the Text Area.  (This will separate any
  initial message from the application engine from the messages that are
  responses to the commands.)
\item Call the {\it isActive} method of every command, and make each menu
  item of each menu active or inactive (greyed out) according to the
  result of the {\it isActive} method of the command.
\end{enumerate}

\section{Selecting a Command from a Menu}

When the user selects a command from a menu:
\begin{enumerate}
\item Set a String field {\tt lastDisplayedString} to the empty string.
  (Rationale:  this will for displaying the last text that the program
  shows the user.)
\item If any table selection is {\it incomplete} (see below) for the command,
  then pop up a dialog box with an ``OK'' button telling the user
  which tables they have to select rows in in order to issue the
  command, and a description of the bounds of the number of rows to
  select (e.g., ``at least 2'', ``between 1 and 5'').
\item Otherwise, if no stage in the command has a queryable parameter
  (i.e., there are no parameters to any of the stages of the command, or all
  parameters in all stages of the command are
  {\tt tableEntry} parameters corresponding to browsable tables):
  \begin{enumerate}
  \item Call {\tt gemSetting.selectCurrentCommand(cmdName)}.
  \item Call {\tt gemSetting.selectCurrentStage(0)}.
  \item For each stage \verb/i/ in the command, perform the Stage Loop
    from the IntI Specification document.
  \item Call the application engine method.
  \item Execute the Command Wrapup Procedure (see below) with parameter
    {\tt false} (indicating command was not cancelled).
  \end{enumerate}
\item Otherwise (i.e., if no table selection is incomplete for
  the command, but there are queryable parameters):
  \begin{enumerate}
  \item Call {\tt gemSetting.selectCurrentCommand(cmdName)}.
  \item Create the Command Dialog for the command.
    (The Command Dialog will take over the rest of the processing
    of the command.)
  \end{enumerate}
\end{enumerate}
A table selection is {\it incomplete for a given command}
if it contains at least one stage with a parameter of type
{\tt tableEntry}, such that the {\tt MinNumberOfReps} for that
parameter is greater than the number of rows that are selected
in the {\tt SourceTable} of the parameter, and the parameter
does not have a {\tt ParentParameter}.

\section{Command Wrapup Procedure}

The Command Wrapup Procedure is called after the user selects a
command from the menu and some processing has been done.  It takes
one parameter, called {\tt commandWasCancelled}.
\begin{enumerate}
\item If there is a Command Dialog, delete it.
\item If {\tt commandWasCancelled} is false (i.e., the application
  engine method corresponding to the command has been called):
  \begin{enumerate}
  \item Determine whether the application should quit, by checking
    the {\tt QuitAfter} attribute and/or calling the {\tt QuitAfterIf}
    method of the command.
  \item If the application should quit:
    \begin{enumerate}
    \item If {\tt lastDisplayedText} is non-null and not the empty string,
      then create a dialog box containing {\tt lastDisplayedText} and
      an ``OK'' button, and display it.  Wait for the user to
      click ``OK'', and then delete the dialog box.
      (Rationale:  if the command executed a {\tt showText} method and
      then Star exits, it may exit before the user has had time to
      read the text.)
    \item Exit the application, e.g. using {\tt System.exit(0)}.
    \end{enumerate}
  \item Otherwise:
    \begin{enumerate}
    \item Append a horizontal line to the Text Display Area, in order to
      separate the last command's output from the output of any
      future commands.
    \item Refresh the tables (see below).
    \end{enumerate}
  \end{enumerate}
\item Call the {\it isActive} method of every command, and make each
  menu item of each menu active or inactive (greyed out) according to
  the result of the {\it isActive} method of the command.
\end{enumerate}

\section{Refreshing the Tables}

To refresh the tables:
\begin{enumerate}
\item For each table currently being displayed in the Table Area
  that is now hidden, delete the corresponding tab.
\item For each table not currently being displayed in the Table
  Area that is now non-hidden, create a tab in the Table Area.
  (After this happens, it should be the case that the only tables
  with tabs in the Table Area are non-hidden tables.)
\item For each non-hidden table which has been updated since the
  last command execution terminated, update the data on the tab
  to reflect the contents of the table.
\item If there is a top table now, place that table's tab on top
  in the tab pane.
\end{enumerate}

\section{The {\tt ShowTextHandler}}

The ShowTextHandler for Star handles text according to the prominence of
the text.
\begin{enumerate}
\item 0-1999:  For each line of text in the message:
  \begin{enumerate}
  \item Truncate the line of text, if necessary, to fit in the Status Bar.
    (Rationale:  it should fit all right, but just in case it doesn't, we
    can show part of it.  Because it is low-priority, it seems OK to
    just show part of it.)
  \item Set the Status Bar to the text.
  \end{enumerate}
  [Note: This will cause the last line of the most recent message to
  overwrite anything that was in the Status Bar before.  This is OK
  because these messages are ``low prominence''.]
\item 2000-2999:  Append the text to the text being displayed in the
  Text Display Area.  Append the text also to {\tt lastDisplayedText}.
\item 3000 and higher:  Create a dialog box containing the text and an
  ``OK'' button, and display it.  When the user clicks the ``OK'' button,
  the box should be deleted.
\end{enumerate}

\section{The Command Dialog Box}

\paragraph{Creating the command dialog box:}
\begin{enumerate}
\item Call {\tt gemSetting.selectCurrentStage(0)}.
\item Perform an Initialize Stage procedure.
\item While the current stage has no queryable parameters, perform
  a Next Stage procedure (see below).
\item Update the dialog box to reflect the current stage, as described
  in the Star GUI Specification document.
\end{enumerate}

\paragraph{Initialize Stage procedure:}
\begin{enumerate}
\item If the current stage is a stage that has never been initialized so far,
  then for each parameter in the current stage:
  \begin{enumerate}
  \item If the parameter has a default value, set the value of the
    parameter in the CommandModel to the default value.
    (Rationale:  since the GUI widgets are linked to the CommandModel,
    this should have the effect of updating the value in the GUI.)
  \item Otherwise, if the parameter has a DefaultValueMethod, then call it
    and set the value of the parameter in the CommandModel to the default
    value.
  \end{enumerate}
\end{enumerate}

\paragraph{Next Stage procedure:}
\begin{enumerate}
\item Perform a Wrap Up Stage procedure.
\item If the Wrap Up Stage procedure returns true, then (assuming
  that the current stage is in the variable {\tt currentStage}):
  \begin{enumerate}
  \item Set the current stage to \verb/currentStage+1/.
  \item Call {\tt gemSetting.selectCurrentStage(currentStage)}.
  \item Perform an Initialize Stage procedure (see below).
  \end{enumerate}
\end{enumerate}

\paragraph{Previous Stage procedure:}
\begin{enumerate}
\item Perform a Wrap Up Stage procedure.
\item If the Wrap Up Stage procedure returns true, then (assuming
  that the current stage is in the variable {\tt currentStage}):
  \begin{enumerate}
  \item Set the current stage to \verb/currentStage-1/.
  \item Call {\tt gemSetting.selectCurrentStage(currentStage)}.
  \item Perform an Initialize Stage procedure (see below).
  \end{enumerate}
\end{enumerate}

\paragraph{Wrap Up Stage procedure:}
\begin{enumerate}
\item Validate the current values of the current repetitions of the
  parameters, as indicated in requirements 26-51 of the IntI
  specification.
\item If any parameter or parameter repetition does not pass validation,
  then:
  \begin{enumerate}
  \item Collect information in a string about anything that does
    not pass validation.
  \item Present that information to the user in a dialog box.
    The dialog box should have just an ``OK'' button.
  \item When the user presses OK, return false.
  \end{enumerate}
\item Load the current values of the current repetitions of the parameters
  for the current stage into the Gem.
\item Call the ParameterCheckMethod of the current stage, if it has one.
\item If the ParameterCheckMethod exists and returns a non-null, non-empty
  string:
  \begin{enumerate}
  \item Show the user the string in a popup with an ``OK'' button.
  \item When the user clicks OK, if the current stage has no queryable
    parameters, then perform the Command Wrapup procedure, with the
    parameter {\tt true}.  (Rationale:  This might happen in some obscure
    situations, such as when a {\tt tableEntry} parameter has a parent
    parameter in another stage which gets set to the parent value.  In
    these situations, there is nothing we can do but cancel the command.)
  \item Return false.
  \end{enumerate}
\item Otherwise, return true.
\end{enumerate}

\paragraph{When the user presses the Next button (if it is not greyed out):}
\begin{enumerate}
\item Perform a Next Stage procedure (see below).
\item While the current stage has no queryable parameters, perform
  a Next Stage procedure (see below).
\item Update the dialog box to reflect the current stage, as described
  in the Star GUI Specification document.
\end{enumerate}

\paragraph{When the user presses the Previous button (if it exists and
  is not greyed out):}
\begin{enumerate}
\item Perform a Previous Stage procedure (see below).
\item While the current stage has no queryable parameters, perform
  a Previous Stage procedure (see below).
\item Update the dialog box to reflect the current stage, as described
  in the Star GUI Specification document.
\end{enumerate}

\paragraph{When the user presses the OK button (if it exists and
  is not greyed out):}
\begin{enumerate}
\item Perform the Wrap Up Stage procedure.
\item For each stage \verb/i/ in the current command, from stage 0 to the
  last stage:
  \begin{enumerate}
  \item Call {\tt gemSetting.selectCurrentStage(i)}.
  \item Perform the Initialize Stage procedure.
  \item Perform the Wrap Up Stage procedure.
  \item If the Wrap Up Stage procedure returns false, then return.
  \end{enumerate}
  (Rationale:  there may have been some previous stages which have
  become invalid as a result of the current stage; also, there may be
  stages with non-queryable parameters whose parameter values have never
  been loaded.)
\item Call the command method.
\item Perform the Command Wrapup procedure, with the parameter false
  (indicating the command was not cancelled).
\end{enumerate}

\paragraph{When the user presses the Cancel button:}
\begin{enumerate}
\item Perform the Command Wrapup procedure, with the parameter {\tt true}.
\end{enumerate}

\section{The Parameter Section}

\paragraph{When the user clicks the Add Another button:}
\begin{itemize}
\item Add another repetition section to the parameter section for the
  parameter, at the bottom of the list.  You may need to add a repetition
  to the model as well.
  (Rationale:  This is safe because, if the Add Another button exists
  and is not greyed out, then we are not at the maximum number of
  repetitions for the parameter yet.)
\end{itemize}

\paragraph{When the user clicks the Move Up button for repetition $k$:}
\begin{itemize}
\item Exchange the value in repetition $k$ with the value in repetition $k-1$.
  This should be done in the model so that the changes will be automatically
  reflected in the GUI.
  (Rationale:  If the Move Up button exists and is not greyed out, then
  there is another repetition above the $k$th repetition.)
\end{itemize}

\paragraph{When the user clicks the Move Down button for repetition $k$:}
\begin{itemize}
\item Exchange the value in repetition $k$ with the value in repetition $k+1$.
  This should be done in the model so that the changes will be automatically
  reflected in the GUI.
  (Rationale:  If the Move Down button exists and is not greyed out, then
  there is another repetition below the $k$th repetition.)
\end{itemize}

\paragraph{When the user clicks the Delete button for repetition $k$:}
\begin{itemize}
\item Delete repetition $k$.  This should be done in the model as well.
  (Rationale:  If the Delete button exists and is not greyed out, then
  we are not at the minimum number of repetitions for the parameter yet.)
\end{itemize}

\end{document}
