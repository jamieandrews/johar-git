\documentclass[11pt]{article}

\setlength{\oddsidemargin}{0in}
\setlength{\evensidemargin}{0in}
\setlength{\topmargin}{-0.5in}
\setlength{\headheight}{0pt}
\setlength{\topskip}{0pt}
\setlength{\textwidth}{6.5in}
\setlength{\textheight}{9in}
%\setlength{\parindent}{0pt}
\setlength{\parskip}{1mm}

%\usepackage{epsf}
\usepackage{graphicx}

% Magic incantation to allow proper PostScript to PDF conversion:
  \usepackage[T1]{fontenc} % type 1 fonts
  \usepackage{times}       % Times-Roman

\newcommand{\Fwname}{Johar}

\sloppy

\begin{document}

\begin{center} \bf \Large
Johar Interface Description File (IDF) \\
Format Specification
\end{center}

This document describes the specification of the Johar Interface
Description File (IDF), version 1.0.

\section{Syntax}

\subsection{Syntax of Attribute Declarations}

An IDF consists of {\it attribute declarations}, possibly
separated by whitespace.  Whitespace has no particular meaning
except inside string or longtext values (see below).
Each attribute declaration is of one of the following forms:
\begin{center}
\begin{tabular}{l}
{\it Attribute} \\
{\it Attribute name} \\
{\it Attribute = value} \\
{\it Attribute name = value} \\
\end{tabular}
\end{center}
Each {\it Attribute} is an upper-case identifier, i.e.\ a
sequence of letters, numbers and underscores starting with an
upper-case letter.  Attributes must be those named in the
specification below.  Each {\it name} is a lower-case
identifier, i.e.\ a sequence of letters, numbers and underscores
starting with a lower-case letter.  Generally, each {\it name}
is chosen by the writer of the IDF.

Values are of four different forms: identifiers, strings,
longtext, and structures.  An {\it identifier} is a
sequence of letters, numbers and underscores.  A {\it string}
is a sequence of characters in-between paired double quote
characters.  Inside a string, the backslash is the escape
character:  the sequence \verb^\"^ indicates a literal double
quote, the sequence \verb^\\^ indicates a literal backslash,
and the backslash followed by a newline indicates a literal
newline.

Although multi-line strings can be constructed by using the
backslash followed by a newline, this is not a very convenient
way of entering them.  A {\it longtext} is a value format
designed for multi-line text.  A longtext value consists of
a sequence of two open curly braces (``\verb/{{/'') at the end
of a line, followed by any lines of arbitrary text, followed by
a sequence of two close curly braces (``\verb/}}/'') on a
separate line.

A {\it structure} consists of a single open curly brace,
followed by any number of attribute declarations, followed
by a single close curly brace.  Each attribute declaration
has the format described at the top of this section.
If an attribute declaration contains a structure, then
the attributes in the structure are referred to as
{\it child} attributes, and the enclosing attribute as the
{\it parent} attribute; we also say that the child attributes
are {\it sub-attributes} of the parent.

\subsection{Example}

\begin{figure}
\begin{verbatim}
Table clientData
Command open = {
  BriefHelp = "Open or create client data file."
  MultiLineHelp = {{
    Open the file containing data about clients.
    If the file does not exist, it is created.
  }}
  Parameter preferredExtension = {
    Type = text
  }
}
\end{verbatim}
\caption{
  Example of IDF syntax.
}
\label{idf-example-fig}
\end{figure}

As an example of all of the above, consider the attribute declarations
shown in Figure \ref{idf-example-fig}.
In that example, there happen to be no repetitions of
attribute names, although in some cases attributes can be
repeated.  The {\tt Table} attribute has a name but no value;
the {\tt Command} and {\tt Parameter} attributes each have both
a name and a value (both values happen to be structures); and
all other attributes have a value but no name.  {\tt Type} has
an identifier value, {\tt BriefHelp} has a string value, and
{\tt MultiLineHelp} has a longtext value.  {\tt BriefHelp},
{\tt MultiLineHelp}, and {\tt Parameter} are sub-attributes of
{\tt Command}, and {\tt Type} is a
sub-attribute of {\tt Parameter}.

\subsection{Processing of Identifiers as Values}

Wherever a double-quoted string literal can appear as a value
for an attribute, an upper- or lower-case identifier can also
appear if such an identifier would be in the right format.
In this case the value is treated as a string
consisting of just the characters in the identifier.  For
instance, the string \verb/"tableEntry"/ and the lower-case
identifier \verb/tableEntry/ are treated exactly the same
when they appear as the value of a parameter.

\section{Allowed Attributes}

In what follows below, if no mention is made of the {\it name}
corresponding to an attribute, then the attribute must have no associated
name.  Each attribute is followed by a description of the
{\it multiplicity} of the attribute, i.e.\ the number of times that the
attribute can appear at the top level and/or as a sub-attribute of another
attribute.  If an attribute is optional (has multiplicity ``0 or 1''),
it may have a {\it default value}, which is also described.

\subsection{Top-level Attributes}

This section describes the attributes that can appear at the top
level of the IDF.  

%%%%%%%%%%%%%%%%%%%%%%%%%%%%%%%%%%%%%%%%%%%%%%%%%%%%%%%%%%%%%%%%%%%%%%%%%%%%

\begin{itemize}

\item \underline{\tt Application}:  The unique
identifier of the application.  (Required)
\begin{itemize}
\item {\it value}: An upper-case identifier (the name of the application).
\end{itemize}
Multiplicity: Exactly 1. \\

\item \underline{\tt ApplicationEngine}:  The class which contains the
application-specific logic of the application.
\begin{itemize}
\item {\it value}: An upper-case identifier (the name of the
  application engine class).
\end{itemize}
Multiplicity: 0 or 1. \\
Default value: The value of the {\tt Application} attribute.

\item \underline{\tt Command}:  A structure representing the basic
top-level unit of user-application interaction.
\begin{itemize}
\item {\it name}: A lower-case identifier (the name of the command).
\item {\it value}: A structure.  See Sections
  \ref{sub-attr-command-sec}, Sub-attributes of {\tt Command},
  and \ref{sub-attr-stage-sec}, Sub-attributes of {\tt Stage} and
  Single-Stage {\tt Command}s.
\end{itemize}
Multiplicity: 1 or more.

\item \underline{\tt CommandGroup}:  A structure representing a
group of related commands.
\begin{itemize}
\item {\it name}: A lower-case identifier (the name of the command group).
\item {\it value}: A structure.  See Section
  \ref{sub-attr-command-group-sec}, Sub-attributes of {\it CommandGroup}.
\end{itemize}
Multiplicity: 0 or more. \\
A {\tt CommandGroup} is a group of conceptually related commands.
The commands themselves are defined in {\tt Command} attributes;
{\tt CommandGroup}s group them by name, and contain no additional
information about the {\tt Command}s.

An interface interpreter may use command group information in order
to structure the interface.  For instance, a classic GUI might
present each command group in a separate dropdown menu on a menu
bar, and any interface interpreter might use command group information
in order to structure help information.

A command cannot appear as a member of more than one command group.
All commands that are not explicitly put into a command group in
an IDF are put into an implicitly-defined command group, whose name
is {\tt commands}.  In particular, if no {\tt CommandGroup}s are
defined, all commands become members of {\tt commands}.

\item \underline{\tt IdfVersion}:  The version of the IDF format used
in the creation of the file.  (Required)
\begin{itemize}
\item {\it value}: A double-quoted string literal, containing a valid
  major and minor version number.
\end{itemize}
Multiplicity: Exactly 1. \\
The version of the first public release of Johar will be 1.0.
Therefore, initially all Johar IDFs should contain the declaration
\verb/IdfVersion = "1.0"/.

\item \underline{\tt InitializationMethod}:  The name of the
application engine method used to initialize the engine.
\begin{itemize}
\item {\it value}: A lower-case identifier (the name of the
  application engine method).
\end{itemize}
Multiplicity: 0 or 1. \\
Default value: {\tt applicationEngineInitialize}.

\item \underline{\tt Table}:  The structure representing a list of
similar entities presented to the user by the application engine.
\begin{itemize}
\item {\it name}: A lower-case identifier (the name of the table).
\item {\it value}: A structure.  See Section
  \ref{sub-attr-table-sec}, Sub-attributes of {\tt Table}.
\end{itemize}
Multiplicity: 0 or more.

\end{itemize}

%%%%%%%%%%%%%%%%%%%%%%%%%%%%%%%%%%%%%%%%%%%%%%%%%%%%%%%%%%%%%%%%%%%%%%%%%%%%

\subsection{Sub-attributes of {\tt Command}}
\label{sub-attr-command-sec}

This section describes the attributes that are acceptable sub-attributes
of any {\tt Command} attribute.  In addition, single-stage {\tt Command}s
(commands that have no {\tt Stage} attribute) can contain any of the
attributes described below in Section \ref{sub-attr-stage-sec},
Sub-attributes of {\tt Stage} and Single-Stage {\tt Command}s.

\begin{itemize}

\item \underline{\tt ActiveIfMethod}:  The name of an application
engine method that
can be called to determine if the user should be able to access
the command.
\begin{itemize}
\item {\it value}: A lower-case identifier.  This is the method
  that will be called to determine if the command is active, i.e.\ if
  the user should be allowed to issue the command.
\end{itemize}
Multiplicity: 0 or 1. \\
If no {\tt ActiveIfMethod} attribute is given, then the command
is always active.

It is recommended that the {\tt ActiveIfMethod} is as efficient
as possible (e.g.\ returning only the value of a field or data
member), because it may be called frequently by the interface
interpreter.

\item \underline{\tt BriefHelp}:  A brief help message describing
the purpose of the command.
\begin{itemize}
\item {\it value}: A string of 30 characters or fewer, containing
  no carriage return.
\end{itemize}
Multiplicity: 0 or 1. \\
Default value: The {\tt Label} of the command, truncated to 30
characters, if needed. \\
This attribute, if given, gives a very brief help message
describing the purpose of the command, suitable for such things
as ToolTips.

\item \underline{\tt CommandMethod}:  The application engine
method that is called to actually carry out the command.
\begin{itemize}
\item {\it value}: A lower-case identifier.  This is the method
  that will be called to carry out the actual command.
\end{itemize}
Multiplicity: 0 or 1. \\
Default value:  The command name.

\item \underline{\tt Label}:  The text describing the {\tt Command}
in the interface.
\begin{itemize}
\item {\it value}: Any string.
\end{itemize}
Multiplicity: 0 or 1. \\
Default value:  Derived from the name of the command using standard
camel-case translation (see Section \ref{camel-case-sec}). \\
The {\tt Label}, if given, is the text that will appear (possibly
with an appended ellipsis, ``\ldots'') on the menu item, button,
or other interface element corresponding to the command.

\item \underline{\tt MultiLineHelp}:  A thorough help message
describing the purpose of the command.
\begin{itemize}
\item {\it value}: A string or multi-line text of any length.
\end{itemize}
Multiplicity: 0 or 1. \\
Default value:  The value of the {\tt OneLineHelp} attribute. \\
This attribute, if given, gives a thorough help message.
IDF writers wanting to give thorough help for each {\it parameter}
of a command should use the {\tt MultiLineHelp} attributes of the
{\it parameters}, as this will promote encapsulation and may
(depending on the interface interpreter) give the user more
control of the volume of help given.

\item \underline{\tt OneLineHelp}:  A one-line help message
describing the purpose of the command.
\begin{itemize}
\item {\it value}: A string of 80 characters or fewer, containing
  no carriage return.
\end{itemize}
Multiplicity: 0 or 1. \\
Default value:  The value of the {\tt BriefHelp} attribute. \\
This attribute, if given, gives a help message describing the purpose of
the command, somewhat longer than the {\tt BriefHelp}.  The
{\tt OneLineHelp} messages may be useful for the interface interpreter to
display in a list.

\item \underline{\tt Prominence}:  An integer describing
how prominently the {\tt Command} should be shown to the user.
\begin{itemize}
\item {\it value}: An integer greater than or equal to 0.  Value ranges:
  \begin{itemize}
  \item 3000 or more:  High prominence.  For instance, in a
    classic GUI, commands
    with prominence 3000 or more might be placed on the screen as
    buttons so that they are as quickly accessible as possible.
  \item 2000-2999:  Normal prominence.
  \item 1000-1999:  Reduced prominence.
  \item 0-999:  Low prominence.  For instance, in a classic GUI,
    commands with prominence 0-999 might be placed on a ``More
    commands'' popup.
  \end{itemize}
\end{itemize}
Multiplicity: 0 or 1. \\
Default value: 2000.

\item \underline{\tt Question}:  A question which may be asked
of the user, given the values of the {\tt Parameter}s and
previous {\tt Question}s in the command.
\begin{itemize}
\item {\it name}: A lower-case identifier (the name of the question).
\item {\it value}: A structure.  See Section
  \ref{sub-attr-question-sec}, Sub-attributes of {\tt Question}.
\end{itemize}
Multiplicity: 0 or more.

\item \underline{\tt QuitAfter}:  An indication of whether the interface
interpreter running the application should always terminate after the
command terminates.
\begin{itemize}
\item {\it value}: A Johar boolean (e.g., \verb/yes/ or \verb/no/).
  See Section \ref{johar-booleans-sec}.
\end{itemize}
Multiplicity: 0 or 1. \\
Default value: \verb/no/. \\
A value of \verb/yes/ means that the interface interpreter should always
quit after execution of the {\tt CommandMethod}.  A value of \verb/no/
means that it should expect more commands, unless there is a
{\tt QuitAfterIfMethod} which returns {\tt true} (see below).

\item \underline{\tt QuitAfterIfMethod}:  An application engine method
that is called to determine whether the interface interpreter running the
application engine should terminate after the command terminates.
\begin{itemize}
\item {\it value}: A lower-case identifier (the method to be
  called to determine quit status).
  The method should return a boolean (true if the application
  should terminate, false otherwise).
\end{itemize}
Multiplicity: 0 or 1. \\
If a {\tt QuitAfterIfMethod} attribute is present, the indicated method
is called after the {\tt CommandMethod} is called.  If the
{\tt QuitAfterIfMethod} returns true, then the interface interpreter
takes this to be an indication that the application should
terminate.

\item \underline{\tt Stage}:  One stage in the processing of the
{\tt Command}.  Each stage may take separate {\tt Parameter}s.
\begin{itemize}
\item {\it name}: A lower-case identifier (the name of the {\tt Stage}).
\item {\it value}: A structure.  See Section
  \ref{sub-attr-stage-sec}, Sub-attributes of {\tt Stage} and
  Single-Stage {\tt Command}s.
\end{itemize}
Multiplicity: 0 or more. \\
A {\tt Command} can be broken up into several {\tt Stage}s.
There can be zero or more explicit {\tt Stage} sub-attributes.
If there are zero explicit {\tt Stage} attributes, then
one stage is implicitly defined.
All the attributes mentioned in Section \ref{sub-attr-stage-sec}
which appear as sub-attributes of the {\tt Command} are then
placed into the one implicitly-defined stage.
The names of the {\tt Parameter}s in all of the {\tt Stage}s in a command
must be disjoint.

\end{itemize}

Each stage in a multi-stage command may be handled at a separate time by
an interface interpreter.  A classic GUI interface interpreter, for
instance, may present a multi-stage command using a ``wizard''-style
dialog box, in which the user can move forward or backward through the
stages by clicking a ``next'' button.  This may facilitate the elicitation
of parameters for infrequently-given commands or commands that have many
parameters.

If a {\tt Command} has more than one {\tt Stage}, then the interface
interpreter may call the {\tt ParameterCheckMethod}s of each stage
separately, and the {\tt DefaultValueMethod}s of each parameter
separately (see below for more thorough information).
This gives a mechanism by which the user can supply values
for parameters in one stage, which are then used to compute the default
values of other parameters.  Thus, if the application programmer needs to
collect user-input values of parameter A (e.g.\ input file name) before
computing the default value of parameter B (e.g.\ output file name),
parameter A can be in an earlier stage and parameter B in a later
stage.

%%%%%%%%%%%%%%%%%%%%%%%%%%%%%%%%%%%%%%%%%%%%%%%%%%%%%%%%%%%%%%%%%%%%%%%%%%%%

\subsection{Sub-attributes of {\tt Stage} and Single-Stage {\tt Command}s}
\label{sub-attr-stage-sec}

These sub-attributes can appear inside a {\tt Stage} in a {\tt Command}.
If the command has no explicit {\tt Stage}s, they can also appear
directly inside the {\tt Command}, in which case they define the
sub-attributes of the one implicit stage of the command.

\begin{itemize}

\item \underline{\tt Parameter}:  A piece of data
which comes from the user and is relevant
to the {\tt Command}, such as an integer, floating-point
number or string.
\begin{itemize}
\item {\it name}: A lower-case identifier (the name of the parameter).
\item {\it value}: A structure.  See Section
  \ref{sub-attr-param-sec}, Sub-attributes of {\tt Parameter}.
\end{itemize}
Multiplicity: 0 or more. \\

\item \underline{\tt ParameterCheckMethod}:  A method in the
application engine that can be
called to check the validity of the stage's parameters.
\begin{itemize}
\item {\it value}: A lower-case identifier.  This is the method
  that will be called to check the validity of the parameter values.
  It should return a string value (null or the empty string if all
  the parameters are valid, an error message if one or more parameters
  are invalid).
\end{itemize}
Multiplicity: 0 or 1. \\
If a {\tt ParameterCheckMethod} is given, then if the user has
input invalid values, the interface interpreter can display the
error message and allow the user to edit the erroneous values
they originally gave.  The {\tt ParameterCheckMethod}
should not do any processing related to the main function of the
command, since the user may later decide to change the parameters, or
even to cancel the command.

If no {\tt ParameterCheckMethod} is given, then the parameter
values will be checked by any implicit rules given for the
parameter (e.g.\ the {\tt MinValue} and {\tt MaxValue} attributes
for an {\tt int} parameter).

\end{itemize}

%%%%%%%%%%%%%%%%%%%%%%%%%%%%%%%%%%%%%%%%%%%%%%%%%%%%%%%%%%%%%%%%%%%%%%%%%%%%

\subsection{Sub-attributes of {\tt Parameter}}
\label{sub-attr-param-sec}

This section describes the attributes that are acceptable
sub-attributes of any {\tt Parameter} attribute.

\begin{itemize}

\item \underline{\tt BriefHelp}:  A brief help message describing
the meaning of the parameter.
\begin{itemize}
\item {\it value}: A string of 30 characters or fewer, containing
  no carriage return.
\end{itemize}
Multiplicity: 0 or 1. \\
Default value: The {\tt Label} of the parameter, truncated to 30
characters, if needed. \\
This attribute, if given, gives a very brief help message
describing the purpose of the parameter.  For instance, a classic
GUI could use this as the text of a ToolTip.

\item \underline{\tt Choices}:  A string representing the possible
choices of values of a parameter of type {\tt choice} (see below,
attribute {\tt Type}).
\begin{itemize}
\item {\it value}:  A double-quoted string literal.
\end{itemize}
Multiplicity:  For parameters of type {\tt choice}, exactly 1.
For other parameters, 0. \\
The string contains the possible choices of values, separated
by bar (``\verb/|/'') characters; e.g. \\
\verb/"portrait|landscape"/,
\verb/"clubs|diamonds|hearts|spades"/.
To include a literal bar character in a choice, the bar should
be preceded by a backslash (``\verb/\/'') character.
To include a literal backslash character in a choice,
use two backslashes (``\verb/\\/'').

\item \underline{\tt DefaultValue}:  The default value for the parameter.
\begin{itemize}
\item {\it value}:  A boolean, integer, real number or double-quoted
  string literal, as appropriate.
\end{itemize}
Multiplicity: 0 or 1. \\
If the user gives no explicit value for a parameter, then the interface
interpreter will act as if they have given the {\tt DefaultValue} as the
value.  For a parameter of type {\tt choice}, the value must be one
of the choices in the {\tt Choices} attribute string.

\item \underline{\tt DefaultValueMethod}:  A method in the application
engine to be called to give the default value for the parameter.
\begin{itemize}
\item {\it value}:  A lower-case identifier (the name of the method
  to be called to return the default value).
\end{itemize}
Multiplicity: 0 or 1.  A parameter cannot have both a {\tt DefaultValue}
attribute and a {\tt DefaultValueMethod} attribute. \\
In a Java application engine, the {\tt DefaultValueMethod} must return a
value of type {\tt boolean},
{\tt long}, {\tt double} or {\tt String}, as appropriate.
For a parameter of type {\tt choice}, the value returned must be one
of the choices in the {\tt Choices} attribute string.

\item \underline{\tt FileConstraint}:  A constraint on the status of
a parameter of type {\tt file} (see below, attribute {\tt Type}).
\begin{itemize}
\item {\it value}: A lower-case identifier.  Accepted values:
  \begin{itemize}
  \item {\tt mustExist}:  The file must exist at the time the
    command is issued.
  \item {\tt mustBeReadable}:  The file must exist and be
    readable by the application at the time the command is issued.
  \item {\tt mustNotExistYet}:  The file must not exist at the
    time the command is issued.
  \item {\tt none}:  No constraint.
% \item {\tt confirmOverwriteIfExists}:  The file will be
%   overwritten, so the interface interpreter must confirm the
%   user's intention to overwrite the file if it currently exists.
  \end{itemize}
\end{itemize}
Multiplicity: For parameters of type {\tt file},  0 or 1.
For other parameters, 0. \\
Default value: {\tt none}.

\item \underline{\tt Label}:  The string that will be used to describe the
parameter if and when the user is asked to enter a value for the parameter.
\begin{itemize}
\item {\it value}:  A double-quoted string literal.
\end{itemize}
Multiplicity: 0 or 1. \\
Default value:  Derived from the name of the command using standard
camel-case translation (see Section \ref{camel-case-sec}). \\
The {\tt Label}, if given, is the text that will appear to the user to
indicate what parameter they are to enter.  For instance, in a classic
GUI, this would be text displayed beside the user-controllable widget used
to set the value of the parameter.

\item \underline{\tt MaxNumberOfChars}:  The maximum number of characters
that can be entered by the user as the value of a {\tt text} parameter
(see below, attribute {\tt Type}).
\begin{itemize}
\item {\it value}:  An integer literal greater than or equal to 1;
  or the lower-case identifier \verb/unlim/.
\end{itemize}
Multiplicity: For parameters of type {\tt text}, 0 or 1.
For other parameters, 0. \\
Default value: {\tt unlim}.

\item \underline{\tt MaxNumberOfLines}:  The maximum number of lines that
can be entered by the user as the value of a {\tt text} parameter
(see below, attribute {\tt Type}).
\begin{itemize}
\item {\it value}:  An integer literal greater than or equal to 1;
  or the lower-case identifier \verb/unlim/.
\end{itemize}
Multiplicity: For parameters of type {\tt text}, 0 or 1.
For other parameters, 0. \\
Default value: 1.

\item \underline{\tt MaxNumberOfReps}:  The maximum number of
repetitions of the parameter that the user can give.
\begin{itemize}
\item {\it value}: An integer literal greater than or equal
  to 1, giving the
  maximum number of repetitions allowed for this parameter;
  or the lower-case identifier \verb/unlim/.
\end{itemize}
Multiplicity: 0 or 1. \\
Default value: 1. \\
The user is not allowed to give more than {\tt MaxNumberOfReps}
repetitions of the parameter.
See {\tt MinNumberOfReps} for more detail.

\item \underline{\tt MaxValue}:  The maximum possible value of
a parameter of type {\tt int} or {\tt float}
(see below, attribute {\tt Type}).
\begin{itemize}
\item {\it value}: An integer or real number literal.
\end{itemize}
Multiplicity:  For parameters of type {\tt int} or {\tt float}, 0 or 1.
For other parameters, 0. \\
Default value:  The maximum value representable in a signed 64-bit integer
(resp.\ 64-bit floating-point) number.

\item \underline{\tt MinNumberOfReps}:  The minimum number of
repetitions of the parameter that the user can give (see below).
\begin{itemize}
\item {\it value}: An integer literal greater than or equal
  to 0, giving the
  minimum number of repetitions allowed for this parameter.
\end{itemize}
Multiplicity: 0 or 1. \\
Default value: 1. \\
{\tt MinNumberOfReps} must be less than or equal to {\tt MaxNumberOfReps}.

The minimum and maximum number of repetitions indicate how many
times the parameter can be repeated.  If a {\tt Parameter}
has a {\tt DefaultValue} or {\tt DefaultValueMethod}, then any
repetitions up to the {\tt MinNumberOfReps} that are not explicitly
changed by the end user are filled in by that value.  If the
{\tt Parameter} has neither a {\tt DefaultValue} nor a
{\tt DefaultValueMethod}, and {\tt MinNumberOfReps} is greater
than 0, then the end user is required to fill in at least
{\tt MinNumberOfReps} repetitions.

For example, a command which takes a person's name as a parameter
might have a {\tt Parameter} of type {\tt text} with no default
value and a minimum and maximum number of repetitions equal to 1,
obliging the user to enter a name.
As another example, a command to show the differences between two files
might take exactly two file parameters.  To enforce this
restriction, the programmer might create a {\tt Parameter} with
type {\tt file} and with {\tt MinNumberOfReps} and
{\tt MaxNumberOfReps} both equal to 2.

\item \underline{\tt MinValue}:  The minimum possible value of
a parameter of type {\tt int} or {\tt float}
(see below, attribute {\tt Type}).
\begin{itemize}
\item {\it value}: An integer or real number literal.
\end{itemize}
Multiplicity:  For parameters of type {\tt int} or {\tt float}, 0 or 1.
For other parameters, 0. \\
Default value:  The minimum value representable in a signed 64-bit integer
(resp.\ 64-bit floating-point) number. \\
{\tt MinValue} must be less than or equal to {\tt MaxValue}.

\item \underline{\tt MultiLineHelp}:  A thorough help message
describing the meaning of the parameter.
\begin{itemize}
\item {\it value}: A string or multi-line text of any length.
\end{itemize}
Multiplicity: 0 or 1. \\
Default value: The {\tt OneLineHelp} of the parameter. \\
This attribute, if given, gives a thorough help message
regarding the parameter.

\item \underline{\tt OneLineHelp}:  A one-line help message describing
the meaning of the parameter.
\begin{itemize}
\item {\it value}: A string of 80 characters or fewer, containing
  no carriage return.
\end{itemize}
Multiplicity: 0 or 1. \\
Default value: The {\tt BriefHelp} of the parameter. \\
This attribute, if given, gives a help message describing the purpose of
the parameter, somewhat longer than the {\tt BriefHelp}.  The
{\tt OneLineHelp} messages may be useful for the interface interpreter to
display in a list.

\item \underline{\tt ParentParameter}:  The name of another parameter
that controls the existence of the current parameter.
\begin{itemize}
\item {\it value}:  A lower-case identifier (the name of the
  parent parameter).
\end{itemize}
Multiplicity: 0 or 1. \\
This attribute and {\tt ParentValue}
are used for parameters that only make sense when some other parameter
has a certain value.  In this case the other parameter is referred
to as the parent parameter.

As an example, when printing a document, the user may choose to
``print to a file'', in which case the user must enter a file name
parameter.  However, if the user does not choose to print to a
file, then there is no reason to expect the user to enter a
file name.  This situation can be set up by giving the
{\tt print} command two parameters:  {\tt printToFile},
a {\tt boolean} parameter, and
{\tt outputFileName}, a {\tt file} parameter
whose parent parameter is {\tt printToFile} and whose
{\tt ParentValue} is \verb/true/.
If the parent parameter does not have the indicated {\tt ParentValue},
then the user is not required to enter any value for the parameter,
regardless of any information about the minimum number of
repetitions required.

Circular {\tt ParentParameter} references are not allowed.  That is,
a chain of {\tt ParentParameter} references must end in a parameter
with no {\tt ParentParameter}.

\item \underline{\tt ParentValue}:  The value of the parent parameter
that triggers the existence of this parameter.
\begin{itemize}
\item {\it value}:  An integer, floating-point or string
  literal, as appropriate.
\end{itemize}
Multiplicity:  For parameters with a {\tt ParentParameter} value,
exactly 1.  For other parameters, 0. \\
See {\tt ParentParameter} for more detail.

\item \underline{\tt Prominence}:  An integer describing how
prominently the {\tt Parameter} should be shown to the user.
\begin{itemize}
\item {\it value}: An integer greater than or equal to 0.  Value ranges:
  \begin{itemize}
  \item 3000 or more:  High prominence.  For instance, in a
    classic GUI, parameters
    with prominence 3000 or more may be placed in the most
    prominent location in the dialog box.
  \item 2000-2999:  Normal prominence.
  \item 1000-1999:  Reduced prominence.
  \item 0-999:  Low prominence.  For instance, in a classic GUI,
    parameters with prominence 0-999 might be accessed only through an
    ``Advanced'' button in the parameter dialog box.
  \end{itemize}
\end{itemize}
Multiplicity: 0 or 1. \\
Default value: 2000.

\item \underline{\tt RepsModel}:  A description of the intended
model for the repetitions of the parameter.
\begin{itemize}
\item {\it value}: A lower-case identifier.  Accepted values:
  \begin{itemize}
  \item {\tt set}:  The repetitions of the parameter are considered
    to be a set of disjoint values.  The interface interpreter does not have
    to keep track of the order in which the user has input the
    values, and does not have to load them in the Gem in the
    order that the user has given them.  The interface interpreter
    does not have to provide a way for the user to input multiple
    repetitions that have the same value.
  \item {\tt multiset}:  The repetitions of the parameter are considered
    to be a set of values, with one repetition possibly having the
    same value as another.  The interface interpreter does not have
    to keep track of the order in which the user has input the
    values, and does not have to load them in the Gem in the
    order that the user has given them.  However, it does have to
    provide the user with the ability to give multiple repetitions
    with the same value.
  \item {\tt sequence}:  The repetitions of the parameter are
    considered to be a sequence of values, with one repetition
    possibly having the same value as another.  The interface
    interpreter must load them in the Gem in the order that
    the user has given them.
  \end{itemize}
\end{itemize}
Multiplicity: 0 or 1. \\
Default value: {\tt set}.

The {\tt RepsModel} may be used by an interface interpreter in order
to structure the interface.  For instance, in a classic GUI, a {\tt choice}
parameter with {\tt MaxNumberOfReps = unlim} and {\tt RepsModel = set} may
be presented as a set of radio buttons, any of which can be turned on.
In contrast, a {\tt file} parameter 
with {\tt MaxNumberOfReps = unlim} and {\tt RepsModel = sequence} must
be presented so that the user can specify a sequence of files, for
instance so that the application engine method is guaranteed to process
them in that order.

\item \underline{\tt SourceTable}:  The {\tt Table} associated
with a {\tt tableEntry} parameter (see below, attribute {\tt Type}).
\begin{itemize}
\item {\it value}: A lower-case identifier (the name of the
  table from which the user selects values for the parameter).
\end{itemize}
Multiplicity: For a parameter of type {\tt TableEntry}, exactly 1.
For other parameters, 0.

\item \underline{\tt Type}:  One of a few values describing
what kind of parameter it is.
\begin{itemize}
\item {\it value}: A lower-case identifier, signifying the
  type of the parameter.  Possible values are:
  \begin{itemize}
  \item {\tt boolean}:  Either true or false.
  \item {\tt choice}:  One of a fixed number of choices.
    Every parameter with a {\tt Type} of {\tt choice} must
    have a {\tt Choices} attribute.
  \item {\tt date}:  A calendar date.
  \item {\tt file}:  A file name.
  \item {\tt float}:  A floating-point number.
  \item {\tt int}:  An integer.
  \item {\tt text}:  A text string.
  \item {\tt tableEntry}:  The parameter is an entry from
    one of the tables declared in the IDF.
    Every parameter with a {\tt Type} of {\tt tableEntry} must
    have a {\tt SourceTable} attribute.
  \item {\tt timeOfDay}:  A time of day.
  \end{itemize}
\end{itemize}
Multiplicity: exactly 1.

\end{itemize}

%%%%%%%%%%%%%%%%%%%%%%%%%%%%%%%%%%%%%%%%%%%%%%%%%%%%%%%%%%%%%%%%%%%%%%%%%%%%

\subsection{Sub-attributes of {\tt Question}}
\label{sub-attr-question-sec}

This section describes the attributes that are acceptable
sub-attributes of any {\tt Question} attribute.

{\tt Question}s are similar to {\tt Parameter}s.  However, the
intention is that the system expects a value for a {\tt Question}
only when an application engine method judges that a value is
required, based on the values of the {\tt Parameter}s.
An interface interpreter may also provide a ``cancel'' option
when asking a question, to allow the user to cancel the command
in response to the question.

\begin{itemize}

\item \underline{\tt AskIfMethod}:  The application engine
method to call in order to see if the question should be asked.
(Required)
\begin{itemize}
\item {\it value}: A lower-case identifier, the name of the
  method to call to determine whether to require a value for
  the {\tt Question}.
\end{itemize}
Multiplicity: exactly 1. \\
The application engine programmer can assume that the values for
the {\tt Parameter}s of the command, and all previous {\tt Question}s,
are accessible from the Gem when the ask-if method is called.

For example, say that the programmer of an editor
application wants the question ``File has been modified.  Save changes?''
to be asked at appropriate points (for instance, for the {\tt close},
{\tt new}  and {\tt quit} commands).  They could do this by adding
the following {\tt Question} to all appropriate commands:
\begin{verbatim}
Question saveIfModified = {
  Type = boolean
  Label = "File has been modified.  Save changes?"
  AskIfMethod = fileModified
}
\end{verbatim}

\item The following attributes of Parameter are also acceptable
  sub-attributes of any {\tt Question}:
  \begin{itemize}
  \item{} {\tt BriefHelp}
  \item{} {\tt Choices}
  \item{} {\tt DefaultValue}
  \item{} {\tt DefaultValueMethod}
  \item{} {\tt FileConstraint}
  \item{} {\tt Label}
  \item{} {\tt MaxNumberOfChars}
  \item{} {\tt MaxNumberOfLines}
  \item{} {\tt MaxValue}
  \item{} {\tt MinValue}
  \item{} {\tt MultiLineHelp}
  \item{} {\tt OneLineHelp}
  \item{} {\tt Prominence}
  \item{} {\tt SourceTable}
  \item{} {\tt Type}
  \end{itemize}

\item The following attributes of Parameter are {\it not} acceptable
  sub-attributes of any {\tt Question}:
  \begin{itemize}
  \item{} {\tt MaxNumberOfReps}
  \item{} {\tt MinNumberOfReps}
  \item{} {\tt ParentParameter}
  \item{} {\tt ParentValue}
  \item{} {\tt RepsModel}
  \end{itemize}
  In addition, the {\tt SourceTable} of any question of type
  {\tt tableEntry} must refer to a non-browsable table.

\end{itemize}

%%%%%%%%%%%%%%%%%%%%%%%%%%%%%%%%%%%%%%%%%%%%%%%%%%%%%%%%%%%%%%%%%%%%%%%%%%%%

\subsection{Sub-attributes of {\tt CommandGroup}}
\label{sub-attr-command-group-sec}

This section describes the attributes that are acceptable
sub-attributes of any {\tt CommandGroup} attribute.

\begin{itemize}

\item \underline{\tt Label}:  The text describing the {\tt CommandGroup}
in the interface.
\begin{itemize}
\item {\it value}: Any string.
\end{itemize}
Multiplicity: 0 or 1. \\
Default value: Derived from the name of the command group using standard
camel-case translation (see Section \ref{camel-case-sec}). \\
The {\tt Label}, if given, is the text that will appear
on the menu heading, button,
or other interface element corresponding to the command group.

\item \underline{\tt Member}:  One of the commands which is
a member of this {\tt CommandGroup}.
\begin{itemize}
\item {\it value}: a lower-case identifier specifying one
  command that is a member of this command group.
\end{itemize}
Multiplicity: 1 or more.

\end{itemize}

%%%%%%%%%%%%%%%%%%%%%%%%%%%%%%%%%%%%%%%%%%%%%%%%%%%%%%%%%%%%%%%%%%%%%%%%%%%%

\subsection{Sub-attributes of {\tt Table}}
\label{sub-attr-table-sec}

This section describes the attributes that are acceptable
sub-attributes of any {\tt Table} attribute.

\begin{itemize}

\item \underline{\tt Browsable}:  A value indicating whether
the user should be able to browse the table.
\begin{itemize}
\item {\it value}: A Johar boolean (e.g., \verb/yes/ or \verb/no/).
  See Section \ref{johar-booleans-sec}.
\end{itemize}
Default value: {\tt yes}.

Some tables are intended to be presented to the user for them
to browse and select rows in, separately from the processing
of a given command.  Others are simply intended to be
tables of candidate values for parameters.  The former kind of
table is referred to as ``browsable'', the latter
``not browsable''.  The {\tt Browsable} attribute controls this
behaviour.

\item \underline{\tt DefaultHeading}:  The default heading of the
table.
\begin{itemize}
\item {\it value}: A double-quoted string or long text.
\end{itemize}
Default value: The {\tt Label} of the table. \\
This attribute gives the heading that will be associated with
the table if no heading is set by the application engine.
If no {\tt DefaultHeading} attribute is given, then the heading will be
derived from the name of the table using standard camel-case
translation (see Section \ref{camel-case-sec}).

\item \underline{\tt Label}:  The text describing the {\tt Table}
in the interface.
\begin{itemize}
\item {\it value}: Any string.
\end{itemize}
Multiplicity: 0 or 1. \\
Default value: Derived from the name of the table using standard
camel-case translation (see Section \ref{camel-case-sec}). \\
The {\tt Label}, if given, is the text that will appear on the menu item,
button, tab, or other interface element corresponding to the table.


\end{itemize}

\subsection{Generated Attribute Values}

When an IDF is read, it is processed into an internal form
that can be used by an interface interpreter.  For convenience,
default values are generated automatically for some sub-attributes if
they are not given explicitly in the IDF.  The descriptions of
these default values are given above, in the sections pertinent
to the individual parameters.

The label and help attributes, if not given, are generated in a specific
sequence.  The sequence, along with the default values generated, are as
follows.
\begin{itemize}
\item {\tt Label}:  The de-camel-cased version of the name of
  the command, command group, parameter or question.
\item {\tt BriefHelp}:  The {\tt Label} for the command, parameter, or
  question, truncated to 30 characters if necessary.
\item {\tt OneLineHelp}:  The value of {\tt BriefHelp}.
\item {\tt MultiLineHelp}:  The value of {\tt OneLineHelp}.
\end{itemize}

\section{Johar Booleans}
\label{johar-booleans-sec}

In some places in Johar IDFs, it is possible to give a boolean value.
In all of these places, the following values are acceptable:
\begin{itemize}
\item \verb/yes/, \verb/Yes/, \verb/YES/, \verb/true/, \verb/True/, or
  \verb/TRUE/, all of which mean the same thing.
\item \verb/no/, \verb/No/, \verb/NO/, \verb/false/, \verb/False/, or
  \verb/FALSE/, all of which mean the same thing.
\end{itemize}
This flexibility is intended to mirror the flexibility which interface
interpreters are encouraged to have in accepting boolean values from end
users of the applications.

\section{Camel Case Translation}
\label{camel-case-sec}

To simplify interface description files, some values of
attributes of commands and parameters are translated into
strings shown to the user by following an algorithm for
translating camel-case identifiers.  This section
describes this algorithm.

``Camel case'' is the phrase used to refer to the practice
of writing identifiers with mixed upper- and lower-case
letters but no underscores.  Often, when camel case
is used, each upper-case letter is intended to start a word.
The algorithm used for camel-case translation is
therefore as follows.
\begin{enumerate}
\item Separate the words in the identifer by single spaces,
  assuming that each upper-case letter starts a new word.
\item Capitalize the first letter of the resulting string
  if it is not capitalized.
\item Translate all the rest of the letters in the resulting string to
  lower-case.
\end{enumerate}

\begin{figure}

\begin{center}
\begin{tabular}{|c|c|}
Camel-Case Identifier & Translation \\
\hline
\verb/add/ & Add \\
\hline
\verb/saveAs/ & Save as \\
\hline
\verb/findInThisPage/ & Find in this page \\
\hline
\verb/inputTextFile/ & Input text file \\
\hline
\verb/inputXMLFile/ & Input x m l file \\
\hline
\verb/inputXmlFile/ & Input xml file \\
\hline
\end{tabular}
\end{center}

\caption{Examples of camel-case translation.}
\label{camel-case-fig}
\end{figure}

Figure \ref{camel-case-fig} shows some examples of the effect of
the camel-case translation algorithm.  The first four
examples are likely to be what the developer wants; the
last two are unlikely to be what the developer wants.
If the effect of the translation algorithm is not what the
developer wants, then they can use the {\tt Label} attribute
to achieve what they want -- for instance, by using a parameter name
{\tt inputXMLFile} with the attribute \verb/Label = "Input XML file"/.

\end{document}

