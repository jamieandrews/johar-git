\documentclass[11pt]{article}

\setlength{\oddsidemargin}{0in}
\setlength{\evensidemargin}{0in}
\setlength{\topmargin}{-0.5in}
\setlength{\headheight}{0pt}
\setlength{\topskip}{0pt}
\setlength{\textwidth}{6.5in}
\setlength{\textheight}{9in}
%\setlength{\parindent}{0pt}
\setlength{\parskip}{1mm}

%\usepackage{epsf}
\usepackage{graphicx}

% Magic incantation to allow proper PostScript to PDF conversion:
  \usepackage[T1]{fontenc} % type 1 fonts
  \usepackage{times}       % Times-Roman

\newcommand{\Fwname}{Johar}

\newcommand{\must}{{\it must}}
\newcommand{\mustnot}{{\it must not}}
\newcommand{\may}{{\it may}}
\newcommand{\doesnothaveto}{{\it does not have to}}
\newcommand{\should}{{\it should}}
\newcommand{\shouldnot}{{\it should not}}

\newcounter{coreReq}

\sloppy

\begin{document}

\begin{center} \bf \Large
Johar Interface Interpreters (IntIs) \\
Requirements Specification for All Java IntIs
\end{center}

This document describes the requirements that must be met by all Johar
interface interpreters (IntIs).  It is assuming that the IntI is written
in Java, but similar requirements apply to IntIs written in any language.

In this document, the words {\must} and {\mustnot} are in italics and
indicate normative statements (strict requirements for the IntI).  The
words {\may} and {\doesnothaveto} are also in italics, but indicate
behaviour that is permitted or not forbidden; they are used for clarity. 
The words {\should} and {\shouldnot} are also in italics, and indicate
behaviour that is recommended but not necessary.

The requirements in this document are numbered 1, 2, 3 and so on.
They will be referred to in other documents as IntI-R1, IntI-R2, IntI-R3,
and so on.

\section{Core Steps}

This section describes the sequence of steps that must be taken by an IntI
in interacting with the IDF, the {\tt GemSetting}, and the application
engine.  We refer to these as the ``core steps''.

\begin{enumerate}
\item A given IntI {\may} take many steps other than the
  core steps, and {\may} interact with the user in many ways outside of
  these core steps (for instance, in providing help or facilitating table
  browsing).
\item However, whenever it interacts with the IDF, the
  {\tt GemSetting} and the application engine, the method calls {\must}
  follow the pattern of the core steps.
\setcounter{coreReq}{\value{enumi}}
\end{enumerate}

\noindent
Core steps:
\begin{enumerate}
\setcounter{enumi}{\value{coreReq}}
\item Initialization phase:
  \begin{enumerate}
  \item The IntI {\must} first get a {\tt johar.idf.Idf} object by using the
    method {\tt johar.idf.Idf.idfFromFile(fname)}, where the file name
    argument is the name of an XML or IDF-format file.
  \item If {\tt johar.idf.Idf.idfFromFile(fname)} throws an
    {\tt IdfFormatException}, then the IntI {\must} show the error message
    and exit.
  \item If the IDF's {\tt IdfVersion} is greater than that supported
    by the IntI, then the IntI {\must} show an error message
    and exit.
  \item The IntI {\must} then call {\tt GemFactory.newGemSetting()},
    using the {\tt Idf} object returned by {\tt idfFromFile} as the
    first parameter and a valid {\it ShowTextHandler} as the second
    parameter.
  \item The IntI {\must} then call {\tt GemSetting.validate()} on the
    {\tt GemSetting} returned by {\tt GemFactory}.
  \item If {\tt GemSetting.validate()} throws an
    {\tt IdfFormatException}, the IntI {\must} show the error message
    and exit.
  \item The IntI {\must} then call {\tt GemSetting.initializeAppEngine()}.
  \end{enumerate}
\item After the initialization phase, the IntI {\may}
  continue with the processing described below.
\item The IntI {\may} then perform the Command Loop
  (see below) as many times as needed, until the IntI terminates.
\item At any time while executing the Command Loop, the IntI
  {\may} exit its current iteration and begin a new iteration.
  [Rationale: the user may cancel the command processing at any time.]
\setcounter{coreReq}{\value{enumi}}
\end{enumerate}

\noindent
The Command Loop consists of the following steps:
\begin{enumerate}
\setcounter{enumi}{\value{coreReq}}
\item The IntI {\may} call the {\tt ActiveIfMethod}s of any commands
  that have them.
\item It {\must} then call
  {\tt GemSetting.selectCurrentCommand(cmdName)}.  If {\tt cmdName}
  corresponds to a command with an {\tt ActiveIfMethod}, then that method
  {\must} be one that was called since the beginning of this iteration of
  the Command Loop.  [Rationale:  the active status of a command may have
  been changed by the effect of the previous command.]
\item It {\must} then call {\tt GemSetting.selectCurrentStage(0)}.
  [Rationale: the parameter values for one stage must be loaded and checked
  before the default values of the parameters in the next stage are
  obtained, to ensure correct communication with the application
  engine methods of multi-stage commands.]
\item The IntI {\may} then perform the Stage Loop (see below)
  as many
  times as needed, at least until {\tt MinNumberOfReps} repetitions of of
  every parameter of every stage of the command has been loaded into the
  {\tt GemSetting}.
\item The IntI {\may} continue to perform the Stage Loop
  after a value for every parameter of every stage of the command has been
  loaded.
\item The IntI {\must} continue to perform the Stage Loop until
  every {\tt ParameterCheckMethod}, in every stage of the command that
  has one, has returned {\tt null} or the empty string.
\item The IntI {\must} perform the following steps
  for every question in the current command,
  from the first question specified to the last question specified.
  \begin{enumerate}
  \item The IntI {\must} call the {\tt AskIfMethod} of the question.
  \item If the {\tt AskIfMethod} returns {\tt true}, and the question
    has a {\tt DefaultValueMethod}, the IntI {\must}
    call the {\tt DefaultValueMethod} of the question.
  \item If the {\tt AskIfMethod} returns {\tt true}, the IntI {\must}
    load a value for the question into the {\tt GemSetting}.
  \end{enumerate}
\item The IntI {\must} call the {\tt CommandMethod} for the
  command.
\item The IntI {\must} ensure that the {\tt showText} text
  from the command has been effectively communicated to the user. 
  [Rationale: the {\tt showText} from the last command executed in a run
  of the application may be important, and must not be rendered
  inaccessible by the termination of the application.]
\item If the command's {\tt QuitAfter} attribute is
  {\tt false}, but the command has a {\tt QuitAfterIfMethod}, then the
  IntI {\must} call that method.
\item If the command's {\tt QuitAfter} attribute is
  {\tt true}, or the command has a {\tt QuitAfterIfMethod} which has
  returned {\tt true}, then the IntI {\must} terminate.
\setcounter{coreReq}{\value{enumi}}
\end{enumerate}

\begin{figure}

\begin{center}
\begin{tabular}{|l|l|l|}
Parameter {\tt Type}	& Java type
	& Comment \\
\hline
{\tt boolean}		& {\tt boolean}
	& \\
\hline
{\tt choice}		& {\tt java.lang.String}
	& One of the choices \\
\hline
{\tt date}		& {\tt java.util.Calendar}
	& \\
\hline
{\tt file}		& {\tt java.io.File}
	& \\
\hline
{\tt float}		& {\tt double}
	& The floating-point type with the maximum \\
 &	& range and precision in Java \\
\hline
{\tt int}		& {\tt long}
	& The integer type with the maximum \\
 &	& range and precision in Java \\
\hline
{\tt text}		& {\tt java.lang.String}
	& \\
\hline
{\tt tableEntry}	& {\tt int}
	& The row number (starting with 0) of one row \\
 &	& that the user has selected \\
\hline
{\tt timeOfDay}		& {\tt java.util.Calendar}
	& \\
\hline
\end{tabular}
\end{center}

\caption{Bindings of Johar parameter types to Java types.}
\label{java-type-binding-figure}
\end{figure}

\noindent
See Table \ref{java-type-binding-figure} for the Java type that a Java
IntI {\must} load as the value of a parameter, depending on what {\tt Type}
the parameter is.

\newpage

\noindent
The Stage Loop consists of the following steps:
\begin{enumerate}
\setcounter{enumi}{\value{coreReq}}
\item At any time, the IntI {\may} call the
  {\tt DefaultValueMethod} of any parameter in the current
  stage.
\item At any time, the IntI {\may} load values for
  repetitions of any parameter in the current stage.
\item If the IntI loads a value for a repetition of a parameter
  with a {\tt DefaultValue} or {\tt DefaultValueMethod}, and the
  value was not selected directly by the user, then the value {\must} come
  from the {\tt DefaultValue} or from a call to the
  {\tt DefaultValueMethod} made since the beginning of this iteration of
  the Stage Loop.  [Rationale: the previous stage's
  {\tt ParameterCheckMethod} may have caused a change to the value
  returned by the parameter's {\tt DefaultValueMethod}.]
\item The IntI {\must} load at least {\tt MinNumberOfReps}
  repetitions of each parameter in the current stage before calling
  the current stage's {\tt ParameterCheckMethod},
  unless the parameter has a {\tt ParentParameter} whose value is
  not the parameter's {\tt ParentValue}.
\item At any time after that, the IntI {\may} call the
  current stage's {\tt ParameterCheckMethod}, if it has one.
\item If the current stage has a {\tt ParameterCheckMethod},
  then the IntI {\must} call it at least once.
\item If the current stage has a {\tt ParameterCheckMethod}, then
  the IntI {\mustnot} load any parameter values between the time
  it last calls the {\tt ParameterCheckMethod} and the time it next
  calls {\tt GemSetting.selectCurrentStage()}.
\item Finally, the IntI {\may} call {\tt GemSetting.selectCurrentStage()}
  with a parameter that is either one greater than the index number of the
  current stage, or less than the index number of the current stage.
\setcounter{coreReq}{\value{enumi}}
\end{enumerate}

\section{Other Requirements}

\noindent Requirements concerning values of parameters and questions:
\begin{enumerate}
\setcounter{enumi}{\value{coreReq}}

\item The IntI {\must} provide a way for the user to select values for
  parameters and questions for any command.

\item For a parameter or question of type {\tt choice}, any value loaded
  by the IntI {\must} be a choice from the parameter's or question's
  {\tt Choices} string.

\item For a parameter or question of type {\tt file}, any value loaded by
  the IntI {\must} respect the parameter's or question's {\tt FileConstraint}.

\item For a parameter or question of type {\tt text}, any value loaded by
  the IntI {\must} respect the parameter's or question's
  {\tt MaxNumberOfChars} and {\tt MaxNumberOfLines} attribute values.

\item For a parameter or question of type {\tt int} or {\tt float}, any
  value loaded by the IntI {\must} respect the parameter's or question's
  {\tt MaxValue} and {\tt MinValue} attribute values.

\setcounter{coreReq}{\value{enumi}}
\end{enumerate}

\noindent Requirements concerning repetitions of parameters:
\begin{enumerate}
\setcounter{enumi}{\value{coreReq}}

\item The IntI {\must} provide a way for the user to select
  multiple values for repetitions of every parameter of a command, up to
  the parameter's {\tt MaxNumberOfReps}.

\item For every parameter, the IntI {\must} load at least
  {\tt MinNumberOfReps} repetitions for the parameter before
  calling the {\tt CommandMethod} of the command, unless the
  parameter has a {\tt ParentParameter} whose value is not
  the {\tt ParentValue} of the parameter.

\item For every parameter, the IntI {\mustnot} load a value
  for the parameter if the parameter has a {\tt ParentParameter} whose
  value is not the {\tt ParentValue} of the parameter.

\item For every parameter, the IntI {\must} load at most
  {\tt MaxNumberOfReps} repetitions for the parameter before
  calling the {\tt CommandMethod} of the command.

\item If the {\tt RepsModel} of a parameter is {\tt set}, then
  the IntI {\may} load only one repetition of each value selected by the user.

\item If the {\tt RepsModel} of a parameter is {\tt multiset} or
  {\tt sequence}, then the IntI {\must} load the number of repetitions
  of each value selected by the user.

\item If the {\tt RepsModel} of a parameter is {\tt set} or
  {\tt multiset}, then the IntI {\may} load values in any order.

\item If the {\tt RepsModel} of a parameter is {\tt sequence},
  then the IntI {\must} load values in the order specified by
  the user.

\setcounter{coreReq}{\value{enumi}}
\end{enumerate}

\noindent Requirements concerning user interface structure:
\begin{enumerate}
\setcounter{enumi}{\value{coreReq}}

\item The IntI {\may} use the value of {\tt Application} as a
  unique identifier of the application currently being run.

\item The IntI {\may} use the application's command groups
  in order to structure the interface.

\item The IntI {\may} use the {\tt Prominence} of commands,
  etc.\ in order to structure the interface.

\item The IntI {\should} use the {\tt Label} of a command, parameter, or
  question as a description, where needed in the interface.

\item The IntI {\may} use the {\tt RepsModel} of any
  parameter in order to structure the interface.

\item The IntI {\should} provide a convenient way for the user
  to select valid values for parameters or questions of type {\tt date},
  {\tt file}, and {\tt timeOfDay}.

\setcounter{coreReq}{\value{enumi}}
\end{enumerate}

\noindent Other IntI requirements:
\begin{enumerate}
\setcounter{enumi}{\value{coreReq}}

\item The IntI {\must} provide access to all commands in the
  application.

\item For a command with any stage with any parameter of type
  {\tt tableEntry}, where the {\tt SourceTable} is browsable,
  the IntI {\mustnot} select the command as the
  current command unless the user has selected at least
  {\tt MinNumberOfReps} entries in the parameter's {\tt SourceTable}.

\item The IntI {\must} allow the user to access, browse and
  select rows in all browsable tables that are not currently hidden.

\item The IntI {\mustnot} allow the user to access, browse
  or select rows in non-browsable tables.

\item The IntI {\mustnot} allow the user to access, browse
  or select rows in tables that are currently hidden.

\item The IntI {\must} provide access to all the help
  messages provided in the IDF.

\item The IntI {\may} use any of the {\tt Label}, {\tt BriefHelp},
  {\tt OneLineHelp}, and {\tt MultiLineHelp} messages provided in the IDF
  wherever they are needed.

\end{enumerate}

\end{document}

