\documentclass[11pt]{report}

\setlength{\oddsidemargin}{0in}
\setlength{\evensidemargin}{0in}
\setlength{\topmargin}{-0.5in}
\setlength{\headheight}{0pt}
\setlength{\topskip}{0pt}
\setlength{\textwidth}{6.5in}
\setlength{\textheight}{9in}
%\setlength{\parindent}{0pt}
\setlength{\parskip}{1mm}

%\usepackage{epsf}
\usepackage{graphicx}

% Magic incantation to allow proper PostScript to PDF conversion:
  \usepackage[T1]{fontenc} % type 1 fonts
  \usepackage{times}       % Times-Roman

\newcommand{\Fwname}{Johar}

\sloppy

\begin{document}

\begin{center} \bf \Large
Johar Application Developers' Guide
\end{center}

\section{Application Design Tasks}

\subsection{Identify Commands}

{\it Commands} are the entry points with which the user is able
to access the facilities of the application.  If you are
familiar with GUI applications, you may find it useful to think
of commands as the menu items within each menu at the top of a
typical application screen (for instance, ``Open'', ``Close'',
``Save'', and ``Save As'' in a typical ``File'' menu).  However,
you should keep in mind that not all interface interpreters will
present commands to users as menu items.

It is often best to identify commands by thinking about what
common tasks users will want to accomplish with your
application, and what steps they will have to take in order to
accomplish them.  There are a lot of tradeoffs and design
decisions to be made here.  A typical tradeoff is between (a)
having a small number of commands, each with many parameters,
that allow the user to accomplish very specific goals with very
few commands, and (b) having a larger number of commands, each
with fewer parameters, that require users to go through more
steps to accomplish their goals but allow them more flexibility.

\subsection{Identify Parameters}

When the user selects a command, they are really just selecting
a single concept, word or phrase that describes in a general way
what they want to do next.  {\it Parameters} are the numeric,
string and other data that goes along with a command and give
more precise details about what the user wants to do.  You will
want to identify what parameters go with your commands.

Typical examples of parameters include:
\begin{itemize}
\item A page number to start printing at;
\item A file to save data to;
\item The width in centimetres of a graphic;
\item The user's given name;
\item Whether to sort data by name, date, or size; and
\item Whether to sort in ascending or descending order.
\end{itemize}
Each parameter has a Johar {\it type}.  The types of the above
six example parameters would probably be
{\tt int},
{\tt file},
{\tt float},
{\tt text},
{\tt choice}, and
{\tt boolean},
respectively.
% might also be tableEntry
The type of a parameter typically determines, in part, how an
interface interpreter allows the user to specify the value for
the parameter.

Each parameter can also have a default value.  You might also
need to decide whether the parameter is optional or not, and
whether the user can give many values for the parameter (for
instance, to give many filenames to process).

\subsection{Identify Selection Data}

For some parameters to your commands, your users will be
selecting one or more items from a list of possible choices.
You will want to identify this selection data.  For instance, as
mentioned above, the user might want to select whether to sort
data by name, date, or size; in this case, ``by name'', ``by
date'' and ``by size'' will be the choices.  As another example,
for an image library application,
the user might want to edit information about an image
they have entered into a database using the application; in this
case, each image previously entered will be one of the choices.

The first question to ask about a kind of selection data is:
For this kind of selection data, will the
user have a fixed set of choices, that never changes except from
one release of the application to the next?  If this is the case,
then it is easiest to code the selection data as a {\tt choice}
parameter.  This will allow you to elicit the information from
the user easily and still update the set of choices when you
release a new version.

For example, say your application is a
sound file converter which can convert to WAV or MP3 format.
Later you might want to add the ability to convert to MP4
format, but for now those are the only two choices the user will
have.  You can code the selection as a {\tt choice} parameter to
some ``Convert'' command, giving ``WAV'' and ``MP4'' as the only
two choices; later you can update the choice list in a new
release.

On the other hand, if, for this selection data, the user will
have a set of choices that may change during the course of a
run, or from one run to the next, it is best to code the
selection data as a Johar {\it table}.  A table is a list of
items that your application can update during the course of a
run.

Tables can be further separated into {\it browsable} and
{\it non-browsable} tables.  Browsable tables are handled
specially by interface interpreters; users are typically able to
examine and select ``rows'' in the table and later specify what
your application is supposed to do with the selected rows.
This is a good choice for tables with a large number of entries,
and where you imagine the user browsing through the entries
in order to decide what to do with them.
Non-browsable tables are not directly accessible to the user,
but the user can choose from the values in the table to fill in
parameter values.  This is a good choice for tables with a small
number of entries, or data that will probably not be the primary
focus of the user's browsing activities.

An example of a browsable table might be a table of images for
the image library application.  The user might have entered a
large number of images, and may have several operations to apply
to them, such as ``delete image'', ``add annotation'', and
``email''.  You might intend in your application for the user to
spend much of their time looking at the image data and deciding
which operation (if any) to apply to each image.

An example of a non-browsable table in the image library
application might be a table of email
addresses of the user's contacts.  While this is not fixed from
run to run, and may even change within the course of one run,
you might decide that for your application it is unlikely that
the user will want to browse the data.  In order to avoid
cluttering the interface unnecessarily, you might therefore
decide to make the table non-browsable, allowing the user to
select email addresses from it only in the context of an
``email'' command.

\subsection{Identify Quit Command}

A small and easy but important task is to identify the command
that the user will issue to quit the system.  If you don't add
any command like this, then...

Hmm... maybe the IntI should automatically add one.

\subsection{Refine Design}

There are many refinements that you may end up making.  Typical xxxxx...

Questions

Staged commands

Parameter value checking

Command groups

\section{Implementation Overview}

\subsection{Setting Up the IDF and Application Engine}

The Interface Description File (IDF) is what allows the user's
interface interpreter to connect with the application.  The
application engine is the class that actually implements the
operations that the user is trying to access in your application.

The initial setup of the IDF and application engine is simple,
and starts with picking an application name.
The easiest way is to pick a descriptive phrase (say ``Image library''),
capitalize each word and run them together to produce the
name (for our example, {\tt ImageLibrary}).  Then create a file
called {\tt ImageLibrary.idf}, and put the following text into it:
\begin{verbatim}
Application = ImageLibrary
\end{verbatim}
Then create a file called {\tt ImageLibrary.java}, and put the
following text into it:
\begin{verbatim}
class ImageLibrary {
}
\end{verbatim}
You now have the skeleton of an IDF and application engine,
which you will fill in in order to elaborate the commands,
parameters and so on of your application.

Actually, the IDF can be named anything
you want, and if you want the application engine class to be
named something different from the application itself, then
you can add a line to the IDF, e.g.
\begin{verbatim}
ApplicationEngine = "org.yoyodyne.ILAppEngine"
\end{verbatim}
However, for clarity and simplicity, we recommend that the file
names, the application name and the class name all correspond.

Johar IDF version number

\subsection{Implementing Commands}

Each command will correspond to an entry in the interface
description file (IDF) and one method in the application
engine class.

It is often easiest to name the command and the method by
``camel-casing'' the phrase for the command as you would want it
to show up in a GUI menu.  To camel-case a phrase, capitalize
each word of the phrase, jam the words all together by
eliminating spaces, and then make the first letter of the phrase
lower-case.

For instance, commands that you would want to show up as
``Open'', ``Save As'', and ``Use Settings As Default'' in a
menu would be named {\tt open}, {\tt saveAs}, and {\tt useSettingsAsDefault}.
If you are using this technique, the commands would be declared
in the IDF as follows:
\begin{verbatim}
Command open = { ... }
Command saveAs = { ... }
Command useSettingsAsDefault = { ... }
\end{verbatim}
where ``...'' is the full description of the command, as
elaborated below.  The declarations of the corresponding Java
methods in the application engine class would be:
\begin{verbatim}
void open(Gem gem) { ... }
void saveAs(Gem gem) { ... }
void useSettingsAsDefault(Gem gem) { ... }
\end{verbatim}
where ``...'' is the source code of the command method.
Note that each of the methods must have a {\tt void} return type,
and must take exactly one parameter, of type {\tt Gem}.

The phrase that corresponds to a command as presented to the user
is called its {\it label}.  If no explicit label is given for a
command, a label is constructed from the command name by doing the
inverse of the camel-casing operation (for instance, deriving
``Use Settings As Default'' from {\tt useSettingsAsDefault}).
However, in some cases
it makes more sense to give an explicit label.
For instance, for the command ``Save As XML'',
if you call the command {\tt saveAsXML}, it will get a default
label of ``Save As X M L''; the spaces in the phrase ``X M L''
appear because the de-camel-casing operation assumes that each
of the letters is supposed to be a separate word.  In this situation,
it is better to explicitly define a label in the IDF, for instance as
follows:
\begin{verbatim}
Command saveAsXML = {
  Label = "Save As XML"
  ...
}
\end{verbatim}

It is also possible to name the Java method corresponding to the
command differently, for instance by declaring the command in the IDF
as follows:
\begin{verbatim}
Command saveAsXML = {
  Label = "Save As XML"
  CommandMethod = saveUsingXmlSaver
  ...
}
\end{verbatim}
This would cause the interface interpreter to call a method
{\tt saveUsingXmlSaver} instead of {\tt saveAsXML} in order to
execute the command.  However, it is recommended that you avoid
this, in order to keep the IDF as simple as possible and the
correspondence between IDF and application engine as straightforward
as possible.

\subsection{Implementing Numeric, Boolean and Text Parameters}

Numeric parameters can be implemented by adding the appropriate
named {\tt Parameter} attribute to the IDF, for example:
\begin{verbatim}
Command print = {
  Parameter numberOfCopies = {
    Type = int
  }
  ...
}
\end{verbatim}
In the corresponding command method in the application engine,
you can access the value of such a parameter by using the
{\tt getIntParameter} method of the {\tt Gem} class.
Every command method takes a parameter of type {\tt Gem},
and one of the uses of it is to get parameter values.
For instance, the command method corresponding to the above
{\tt print} command might start like this:
\begin{verbatim}
void print(Gem gem) {
  int numberOfCopies = gem.getIntParameter("numberOfCopies");
  ...
}
\end{verbatim}

Similarly, you specify a boolean parameter as type {\tt boolean}
and use method {\tt Gem.getBooleanParameter} to access it;
you specify a floating-point parameter as type {\tt float}
and use method {\tt Gem.getFloatParameter} to access it; and
you specify a text parameter as type {\tt text}
and use method {\tt Gem.getStringParameter}
to access it.  (Note that the method for the {\tt text} type is
{\tt getStringParameter}, not {\tt getTextParameter}; more on
this later.)

For integer and floating-point parameters, you can specify upper
and lower bounds for the values to be entered, for example:
\begin{verbatim}
  Parameter numberOfCopies = {
    Type = int
    MinValue = 1
    MaxValue = 99
  }
\end{verbatim}
One of the tasks of the user's interface interpreter is to
check these bounds if they appear in the IDF, so the application
engine can assume that any value it receives will be within the
bounds.

Similarly, you can specify the maximum number of characters
({\tt MaxNumberOfChars}) and the maximum number of lines
({\tt MaxNumberOfLines}) that can appear in a text parameter.
By default, the maximum number of characters is unlimited, but
the maximum number of lines is 1.  You can set the maximum
number of lines to either an integer or the word {\tt unlim},
meaning ``unlimited''.  Again the interface interpreter is
supposed to check these bounds, so you don't have to.

As with commands, parameters can be given a {\tt Label}
separate from the name of the parameter as it appears in the
IDF.  However, for simplicity and readability, this is not
recommended.  It's therefore important to think carefully about
exactly what a parameter signifies before choosing a name for it.

\subsection{Implementing Date and File Parameters}

date

file

file constraints

\subsection{Default Values, Optional Parameters and Parameter Repetitions}

You can specify a default value for a parameter in the IDF with
the {\tt DefaultValue} attribute; for instance,
\begin{verbatim}
Command print = {
  Parameter numberOfCopies = {
    Type = int
    DefaultValue = 1
  }
  ...
}
\end{verbatim}
The interface interpreter will fill in the default value if the
user does not explicitly select a different value.

If the default value can change over the course of a run or from
one run to the next, then what you probably want is to code a
Java method that returns the default value.  You can do this
by specifying a {\tt DefaultValueMethod}, for instance:
\begin{verbatim}
  Parameter userName = {
    Type = string
    DefaultValueMethod = previouslyUsedUserName
  }
\end{verbatim}
Here, {\tt previouslyUsedUserName} must correspond to a method in
the Java application engine class declared as follows:
\begin{verbatim}
String previouslyUsedUserName(Gem gem) {...}
\end{verbatim}
Note that this is declared identically to a command method, except
that it has return type {\tt String} rather than {\tt void}.
Default value methods for other types of parameters must return
the appropriate type ({\tt int}, {\tt float}, {\tt Date}, {\tt File}, etc.).

If you identify a parameter as {\tt Optional}, this means that
the user doesn't have to give a value for it.  More generally,
you can give information in the IDF that allows the user to
enter zero values, one single value, or more than one value for a parameter.
We refer to these as ``repetitions'' of the parameter.
The attribute {\tt MinNumberOfReps} gives the minimum number of
repetitions that the user can enter, and 
the attribute {\tt MaxNumberOfReps} gives the maximum number.
{\tt MinNumberOfReps} can be any non-negative integer, and
{\tt MaxNumberOfReps} can be any positive integer or the word
{\tt unlim} (meaning ``unlimited'').  If these attributes are
not given for a parameter,
both default to 1, meaning that the parameter works as ``normal'';
that is, the user is expected to enter a single value (or let the
default value be used).
Specifying a parameter as {\tt optional} is
the same  as saying that the minimum number of repetitions is 0
and the maximum is 1.

For instance, consider the following parameter declarations (not
necessarily in the same command):
\begin{verbatim}
  Parameter heading = {
    Type = string
    Optional
  }
  Parameter filesToProcess = {
    Type = file
    MaxNumberOfReps = unlim
  }
  Parameter filesToCompare = {
    Type = file
    MinNumberOfReps = 2
    MaxNumberOfReps = 2
  }
\end{verbatim}
The user can enter either 0 or 1 repetitions of the first parameter,
and 1 or more of the second parameter, but must enter exactly
2 of the third parameter.

The command method can access information about the number of
parameter repetitions by calling the {\tt Gem} methods ...  XXX
In fact, the get... XXX  methods are set up so that ......  XXX
(exceptions)

Note the difference between a parameter with a default value
and an optional parameter.  If you declare a parameter as
{\tt Optional} (or if you declare it as having
{\tt MinNumberOfReps = 0}), then calling XXXX without checking
the number of repetitions provided will result in an exception
being thrown.  This is the case whether or not you provide a
default value for the parameter.  If you provide a default
value but the minimum number of repetitions is 1, then the
interface interpreter will fill in the default value and you
are guaranteed to be able to call XXXX at least once.  In fact,
the IntI is required to fill in the default value up to the
minimum number of repetitions; of course, it also allows the
user to select a different value if they want.

\subsection{Implementing Value-Linked Parameters}

Sometimes, giving a value for one parameter makes sense only
if another parameter has a particular value.  For instance,
one might have a boolean parameter to a ``print'' command called
{\tt printToFile}; if the value of this parameter is {\tt true},
then we would want the user to specify a file name to print to
via another parameter, say {\tt fileToPrintTo}, but otherwise
it makes no sense for the user to specify a value for
{\tt fileToPrintTo}.

In such a situation.... XXXX

\subsection{Implementing Selection Data}

Choice parameter

Must turn tables visible/invisible to indicate no current data

\subsection{Implementing Questions}

\subsection{Adding Help}

\subsection{Other Refinements}

Prominence

Staged commands

Checking Parameter Values

Active and inactive commands

Quit after, if...

\section{Recipe Book}

\end{document}

